\documentclass[12pt, a4paper]{article}

\usepackage{geometry}
\usepackage{amsmath}
\usepackage{amssymb}
\usepackage{fancyhdr}
\usepackage{pgfplots}
\usepackage{cancel}
\usepackage{scalerel}
\usepgfplotslibrary{fillbetween}
\usepackage{comment}

\setlength{\headheight}{50pt}
\addtolength{\topmargin}{-29.68335pt}
\pgfplotsset{
    compat=1.18,
    standard/.style={
    font=\tiny,
    axis line style = thick,
    trig format=rad,
    enlargelimits,
    axis x line=middle,
    axis y line=middle,
    enlarge x limits=0.00005,
    enlarge y limits=0.00005,
    every axis x label/.style={at={(ticklabel* cs:1.05)},anchor=north west},
    every axis y label/.style={at={(ticklabel* cs:1.05)},anchor=south east}
    }
}

\pagestyle{fancy}
\renewcommand{\headrulewidth}{0pt}
\renewcommand{\footrulewidth}{0pt}

\author{Kalle Alkula}
\title{Calculus.vk37}
\date\today
\fancyhf{}
\rhead{
    PINFOSS 22\\
    Kalle.alkula@turkuamk.fi
}
\rfoot{page \thepage}
\begin{document}
\maketitle
\thispagestyle{fancy}
    \paragraph*{9.}
        \textit{
            Funkiton g(x) määritellään asettamalla
        }
    \begin{equation*}
        g(x) =
        \begin{cases}
            x^2 -2, x <0\\
            3x + b, x\geq 0
        \end{cases}.
    \end{equation*}
    \textit{
        Millä parametrin b arvolla funktio on jatkuva pisteessä x = 0? piirrä tätä arvoa vastaava funkiton kuvaaja.
    }
    \\
    \begin{equation*}
        \begin{split}
            \lim_{x \rightarrow 0-} &=~x^2 -2\\
            &=~0^2-2\\
            &=~-2\\
            &=~b = -2\\
            \\
            \lim_{x \rightarrow 0+} &=~3x + b \\
            &=~3(0)+ (-2)\\
            &=~ -2 \\\\
            g(x) &= 
        \begin{cases}
            x^2 -2, x <0\\
            \cancel{3x + b, x\geq 0}\\
        \end{cases}
        \end{split}
    \end{equation*}
    \\
    \begin{align*}
    \begin{tikzpicture}[scale=2]
        \begin{axis}[standard,
            xtick={-5,-4,-3,-2,-1,0,1,2,3,4,5},
            ytick={-5,-4,-3,-2,-1,0,1,2,3,4,5},
            samples=1000,
            xlabel={$x$},
            ylabel={$y$},
            xmin=-5.1,xmax=5.1,
            ymin=-5.1,ymax=5.1,
        ]
        \addplot[axis line style = regular, red]
        {x^2 -2};
        \end{axis}
    \end{tikzpicture}
    \end{align*}
    \newpage
    \paragraph*{10.}
        \textit{
        Funkiton h(x) määritellään
        }
        \begin{equation*}
            \begin{cases}
                h(x)= x+ 1,\neq 0\\
                h(0)= 2
            \end{cases}
        \end{equation*}
        \textit{
            Onko funktiolla epäjatkuvuuskohtia? piirrä kuvaaja.
        }
        \medskip
        \begin{align*}
            \begin{tikzpicture}[scale=1.2]
                \begin{axis}[standard,
                    xtick={-5,-4,-3,-2,-1,0,1,2,3,4,5},
                    ytick={-5,-4,-3,-2,-1,0,1,2,3,4,5},
                    samples=1000,
                    xlabel={$x$},
                    ylabel={$y$},
                    xmin=-5.1,xmax=5.1,
                    ymin=-5.1,ymax=5.1,
                ]
                \addplot[axis line style = regular, red]
                {2};
                \addplot[axis line style = regular, blue]
                {x + 1};
                \end{axis}
            \end{tikzpicture}
            \end{align*}
            on jatkuva pisteessä x= 1
    \newpage
    \paragraph*{101.}
            \textit{
            Tutki Jatkuvuutta.
            }
            \begin{equation*}
                \begin{split}
                \begin{cases}
                    2x + 3,&~x< -2\\
                    -1,&~x= 2\\
                    \frac{1}{2}x,&~x> -2
                \end{cases}
                \end{split}
            \end{equation*}
            \\
            \begin{tikzpicture}[scale=1.5]
                    \begin{axis}[standard,
                        xtick={-5,-4,-3,-2,-1,0,1,2,3,4,5},
                        ytick={-5,-4,-3,-2,-1,0,1,2,3,4,5},
                        samples=1000,
                        xlabel={$x$},
                        ylabel={$y$},
                        xmin=-5.1,xmax=5.1,
                        ymin=-5.1,ymax=5.1,
                    ]
                    \addplot[axis line style = regular, red]
                    {2*x+3};
                    \addplot[axis line style = regular, blue]
                    {-1};
                    \addplot[axis line style = regular, blue]
                    {0.5*x};
                    \end{axis}
                \end{tikzpicture}
                \\
                on jatkuva pisteessä x= -2
    \newpage
    \paragraph*{15.}
    \textit{
        Derivoi seuraava funktio\\
    }
    \textbf{a)}
    \begin{equation*}
    \begin{split}
        f(x)&=-6\\
        f'(x)&=D-6\\
        f'(x)&= 0\\
    \end{split}
    \end{equation*}
    \\
    \textbf{b)}
    \begin{equation*}
    \begin{split}
        h(x)&=x^{100}\\
        h'(x)&=Dx^{100}\\
        h'(x)&=100x^{100-1}\\
        h'(x)&=100x^{99}\\
    \end{split}
    \end{equation*}
    \\
    \textbf{c)}
    \begin{equation*}
    \begin{split}
        g(x)&= \frac{1}{x^3}\\
        g'(x)&= Dx^{-3}\\
        g'(x)&= -3x^{-3-1}\\
        g'(x)&= -3x^{-4}\\
    \end{split}
    \end{equation*}
    \\
    \textbf{d)}
    \begin{equation*}
    \begin{split}
        f(t)&= \frac{1}{\sqrt{t}}\\
        f'(t)&= \sqrt{t}^{-1}\\
        f'(t)&= \frac{1}{2}t^{-\frac{1}{2}-1}\\
        f'(t)&= \frac{1}{2}t^{-\frac{3}{2}}\\
    \end{split}
    \end{equation*}
\newpage
\paragraph*{16.}
    \textit{
        määritä ja sievennä derivaatan f'(x) lauseke kun,
    }
    \\
    \textbf{a)}
    \begin{equation*}
        \begin{split}
            f(x)&=rx^3\\
            f'(x)&=Drx^3\\
            f'(x)&=3rx^2\\
        \end{split}
    \end{equation*}
    \textbf{b)}
    \begin{equation*}
        \begin{split}
            f(x)&=\frac{2}{x^4}\\
            f'(x)&=D\frac{2}{x^4}\\
            f'(x)&=-\frac{2\times 4}{x^{4-1+2}}\\
            f'(x)&=-\frac{8}{x^{5}}\\
        \end{split}
    \end{equation*}
    !!!sain laskettua mutta en ymmärrä täysin!!!\\\
    \textbf{c)}
    \begin{equation*}
        \begin{split}
            f(x)&=2x^4-6x^2+2x-4\\
            f'(x)&=2x^4-6x^2+2x-4\\
            f'(x)&=D(2x^4)D(-6x^2)D(+2x)D(-4)\\
            f'(x)&=8x^3-12x+2+0\\
            f'(x)&=8x^3-12x+2\\
        \end{split}
    \end{equation*}
    \textbf{d)}
    \begin{equation*}
        \begin{split}
            f(x)&=\frac{-4x^2-6x+2}{x^2}\\
            f'(x)&=\frac{D(-4x^2-6x+2)x^2-(-4x^2-6x+2)Dx^2}{(x^2)^2}\\
            \frac{f'(x)}{g'(x)}&=\frac{(-8x-6+2)x^2-(-4x^2-6x+2)2x}{(x^2)^2}\\
            \frac{f'(x)}{g'(x)}&=-\frac{2(-3x+2)}{x^3}
        \end{split}
    \end{equation*}
    hirvee taistelu välivaiheissa
    \newpage
    \paragraph*{17.}
    \begin{equation*}
        \begin{split}
            f(x)&=
        \end{split}
    \end{equation*}


\end{document}